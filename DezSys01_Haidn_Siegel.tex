\documentclass[a4paper]{article}
\usepackage[english]{babel}
\usepackage[utf8x]{inputenc}
\usepackage{amsmath}
\usepackage{graphicx}
\usepackage[colorinlistoftodos]{todonotes}
\usepackage{listings}
\usepackage{glossaries}
\usepackage[
top    = 2.75cm,
bottom = 2.00cm,
left   = 2.50cm,
right  = 2.00cm]{geometry}

\title{Remoting Patterns}

\author{Haidn, Siegel}

\date{\today}

\begin{document}
\maketitle
\newpage

\tableofcontents
\newpage

\section{Aufgabenstellung}
Gruppenarbeit: 2 Mitglieder (Server/Client)\\
\\
Analysieren Sie in einer Gruppe von 2 Leuten die mitgelieferte Implementation der verteilten LeelaApplikation. Identifizieren Sie dabei alle verwendeten Elemente der "Basic Remoting Patterns" und erstellen Sie UML-Klassendiagramme für die Pakete comm, comm.socket, comm.soap, evs2009 und evs2009.mapping.\\
Schließen Sie die unfertigen Tests ab, und dokumentieren Sie etwaige Schwierigkeiten.\\
\\
\textbf{Was ist zu tun?}
\begin{itemize}
	\item UML Klassendiagramm
	\item Erweitern der Testfälle (mind. einen Testfall erweitern)
	\item Kritik und Verbesserungsvorschläge
\end{itemize}
\mbox{} \\
\textbf{Punkte (16):}\\
Identifikation von Basic Remoting Patterns ... 1Pkt\\
Beschreibung der Applikation ... 4Pkt\\
UML-Diagramme ... 3Pkt\\
Schreiben von einem neuen Testfall ... 2Pkt\\
konstruktive Verbesserungsvorschläge / Kritikpunkte ... 6Pkt\\
\\
\textbf{Main-Method-Classes}\\
\begin{center}
	\begin{tabular}{ | r | r |}
		\hline
		SOAPPluginServer & src/main/java/comm/soap/SOAPPluginServer.java \\
		\hline
		Application & src/main/java/evs2009/Application.java \\
		\hline
	\end{tabular}
\end{center}

\newpage
\section{Einleitung}
Design Patterns haben in den letzten Jahren eine bedeutende Rolle in der objektorientierten\\
Softwareentwicklung bekommen.\\
Ein Pattern ist mit einer Drei-Punkte-Regel zu vergleichen und setzt sich aus den folgenden Komponenten zusammen:
\begin{itemize}
	\item Kontext
	\item Problem
	\item Lösung
\end{itemize}
Ziel und Nutzen eines Patterns soll es sein Software zu entwickeln, der es möglich ist sich soweit selbst zu konfigurieren,
dass interne Prozesse auf ein sich änderndes Umfeld angepasst und optimiert werden können.
\section{UML}
	\subsection{comm}
	\subsection{comm.socket}
	\subsection{comm.soap}
	\subsection{evs2009}
	\subsection{evs2009.mapping}
\section{Testfälle}
\section{Kritik und verbesserungsvorschläge}
		
\end{document}