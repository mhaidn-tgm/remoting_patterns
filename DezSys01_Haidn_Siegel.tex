\documentclass[a4paper]{article}

\usepackage[english]{babel}
\usepackage[utf8x]{inputenc}
\usepackage{amsmath}
\usepackage{graphicx}
\usepackage[colorinlistoftodos]{todonotes}
\usepackage{listings}
\usepackage{glossaries}
\usepackage[
top    = 2.75cm,
bottom = 2.00cm,
left   = 2.50cm,
right  = 2.00cm]{geometry}

\title{Remoting Patterns}

\author{Haidn, Siegel}

\date{\today}

\begin{document}
\maketitle
\newpage
\tableofcontents
\newpage
\section{Arbeitszeit}
\subsection{Schaetzung}

\begin{center}
  \begin{tabular}{ | c | c | c | }
    \hline
\textbf{Aufgabe} & \textbf{Aufwand in Stunden} & \textbf{Person} \\ 
    \hline 
    \hline
	Ausfuehren des Projektes & 2 & Haidn \& Siegel \\ 
    \hline
	Identifikation der Basic Remoting Patterns & 1.5 & Haidn \& Siegel \\ 
    \hline
	Beschreibung der Applikation & 2 & Haidn \& Siegel \\ 
    \hline
	UML-Diagramme & 1.5 & Haidn \& Siegel \\ 
    \hline
	Schreiben von einem neuen Testfall & 1 & Haidn \& Siegel \\ 
    \hline
	konstruktive Verbesserungsvorschläge / Kritikpunkte & 1.5 & Haidn \& Siegel \\ 
    \hline
	 \hline
	 \textbf{Gesammt} & Haidn & xxx \\ 
	 \hline
	 \textbf{Gesammt} & Siegel & xxx\\ 
	 \hline
	 \textbf{Gesammt} & Team & xxx \\ 
  \end{tabular}
\end{center}
\subsection{Endzeitaufteilung}

\begin{center}
  \begin{tabular}{ | c | c | c | }
    \hline
\textbf{Aufgabe} & \textbf{Aufwand in Stunden} & \textbf{Person} \\ 
    \hline 
    \hline
	Ausfuehren des Projektes & 2 & Haidn \& Siegel \\ 
    \hline
	Identifikation der Basic Remoting Patterns & 1.5 & Haidn \& Siegel \\ 
    \hline
	Beschreibung der Applikation & 2 & Haidn \& Siegel \\ 
    \hline
	UML-Diagramme & 1.5 & Haidn \& Siegel \\ 
    \hline
	Schreiben von einem neuen Testfall & 1 & Haidn \& Siegel \\ 
    \hline
	konstruktive Verbesserungsvorschläge / Kritikpunkte & 1.5 & Haidn \& Siegel \\ 
    \hline
	 \hline
	 \textbf{Gesammt} & Haidn & xxx \\ 
	 \hline
	 \textbf{Gesammt} & Siegel & xxx\\ 
	 \hline
	 \textbf{Gesammt} & Team & xxx \\ 
  \end{tabular}
\end{center}
\newpage
\section{Ausfuehren des Projektes}	
	%starting at 9:50
	
\end{document}